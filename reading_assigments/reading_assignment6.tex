\documentclass{article}
\usepackage[utf8]{inputenc}
\usepackage{amsmath}
\usepackage{graphicx}
\usepackage{float}
\usepackage[font=scriptsize,labelfont=bf]{caption}

\title{COMP6231 : Reading Assignment 6}
\author{Thanakorn Panyapiang\\
(31446612, tp2n19@soton.ac.uk)}
\date{}

\begin{document}
\maketitle

The Society of Mind theory proposes the idea that mind is a society of agents where each agent is specialized in a specific domain. However, none of an individual agent is capable of solving even the simplest common-sense problem by itself. The ability to solve issues comes from a community of agents collaborate with each other.\\
\indent The theory looks plausible in several contexts especially in nature as the number of cells in the organism is directly proportionate with its abilities. For instance, a unicellular organism, namely bacteria and yeast, has a limited ability that only is necessary for surviving, while a multicellular organism such as plants, animals, and humans, which are composed of hundreds of millions of cells, are able to do more complicated tasks like communicating, adjusting to new environment, and learning new things. Moreover, from biological knowledge, it has been found that cells in multicellular organisms communicate using hormones and chemicals to send signals to each other. This is similar to the idea of K-lines and connection lines which Minsky mentioned in his book.

\end{document}
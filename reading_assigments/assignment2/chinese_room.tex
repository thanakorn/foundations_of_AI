\documentclass{article}
\usepackage[utf8]{inputenc}
\usepackage{amsmath}

\title{COMP6231 : Reading Assigment 2}
\author{Thanakorn Panyapiang(31446612)}

\begin{document}
\maketitle

The Chinese room experiment opposes the idea of the strong AI by focusing on the way a machine finds the result instead of the output it produces(which is proposed in the Turing test).  Searle states that what the man in the room does is only a symbol processing and cannot compare with human thinking.

One observation I have on this argument is that although the person in the room only processes symbols, manipulating symbols is also a part of the thought process of a native Chinese speaker he converses with. When the native speaker gets the answer, he has to correlate a sequence of characters to words he knows and interprets the structure of the sentence based on the language syntax before he can understand the true meaning of the text. In this context, the vocabulary knowledge and the grammatical rules are the program of the native speaker. 

What different between the native Chinese speaker and the man in the room is the native speaker has other programs working on reasoning and analyzing after the first program finish. By combining several programs together, it helps the native speaker to understand the story better.
\end{document}
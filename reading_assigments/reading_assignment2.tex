\documentclass{article}
\usepackage[utf8]{inputenc}
\usepackage{amsmath}

\title{COMP6231 : Reading Assigment 2}
\author{Thanakorn Panyapiang(31446612)}

\begin{document}

\maketitle

Apart from describing the works on the Stanford Cart and the CMU Rover, another interesting topic the paper mentions is general artificial intelligence. The author believes that incrementally solving control and perception challenges of developing a mobile autonomous machine is the most certain way to succeed in developing general AI. He believes it can be compared with the evolution of animals that have adopted a mobile way of life which slowly developed their competence over a long period of time.

The analogy the author makes quite interesting and, with careful consideration, there are several things in common between animals and machine evolutions than it first appears. For example, the CMU Rover uses an omnidirectional wheel which is a newer technology so the robot has more flexibility in movement. This is similar to octopus developed their dexterous manipulators for catching their prey.

However, there is a big difference in the cause of evolution that the author didn't indicate. Evolution in animals occurs because of necessity in surviving while evolution in machines is created by humans. As the cause of evolution is different, the result of evolution over a long period of time could be different.

\end{document}
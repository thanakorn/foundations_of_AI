\documentclass{article}
\usepackage[utf8]{inputenc}
\usepackage{amsmath}
\usepackage{graphicx}
\usepackage{float}
\usepackage[font=scriptsize,labelfont=bf]{caption}

\title{COMP6231 : Reading Assignment 8}
\author{Thanakorn Panyapiang\\
(31446612, tp2n19@soton.ac.uk)}
\date{}

\begin{document}
\maketitle

"The Architecture of Mind: A Connectionist Approach" proposes the theory that the mind is connections of small processing units communicates by passing numbers to each other along the connection lines which link them together.\\
\indent The strength that the connectionist approach has over the classical artificial intelligence approach, in my opinion, is the utilization of parallelism. The conventional AI puts the emphasis on symbolic reasoning and processing. The big issue of this view is it requires the computational unit to be remarkable fast (since it has to deal with many problems at a time) which is biologically impossible based on studies about the human brain. \\
\indent The parallel computation idea proposed by Rumelhart fits better with human intelligence. The human brain is not so fast on symbol manipulations such as performing arithmetic operations. However, it is capable of controlling many organisms in the body simultaneously and efficiently. These capabilities are very challenging for powerful machines. For instance, to build a humanoid robot that can walk smoothly, it requires high-perfomance computers and microprocessors. Since the brain works very slow compared to a computer, the only possible explanation remains for these capabilities is it has an astonishing parallel computing mechanism.

\end{document}
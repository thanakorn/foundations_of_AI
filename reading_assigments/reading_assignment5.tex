\documentclass{article}
\usepackage[utf8]{inputenc}
\usepackage{amsmath}
\usepackage{graphicx}
\usepackage{float}
\usepackage[font=scriptsize,labelfont=bf]{caption}

\title{COMP6231 : Reading Assignment 5}
\author{Thanakorn Panyapiang\\
(31446612, tp2n19@soton.ac.uk)}
\date{}

\begin{document}
\maketitle

The goal of the CYC project is to teach computer a common sense knowledge as it plays a key role in communication and understanding. The founder of the project believes that before any artificial intelligence system cannot be entrusted to handle critical tasks until it manage to acquire this knowledge. 

\indent Although there is no argument about the importance of common sense, identifying what is common sense is subjective and the article does not clarify how the CYC team deals with it.There is no universal common knowledge that has been widely accepted as well as there are no two human beings who have the same thoughts. Our common sense is developed by the environment that we grew up with, the people that we were surrounded, and the things that we value. Things that are common for one person may not be so common for another and vice versa. Let's take one example in the article to illustrate. One application of CYC that the author gives in the article is detecting common-sense errors in the data. For example, if a person fill the gender of his spouse similar to himself in the spreadsheet, the program can correct the information automatically. However, this common-sense is less valid nowadays because there are many people do not care about gender when they get married. The knowledge about gender and marriage has become a personal preference rather than common sense.

\end{document}
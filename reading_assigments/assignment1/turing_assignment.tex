\documentclass{article}
\usepackage[utf8]{inputenc}
\usepackage{amsmath}

\title{COMP6231 : Reading Assigment 1}
\author{Thanakorn Panyapiang(31446612)}

\begin{document}

\maketitle
\section*{Is the Turing Test a way to dodge the issue of what intelligence is altogether?}

Although the Turing Test proposes a systematic way to find the answer to the question, "Can machines think?", the proposed method still depends significantly on the intellectual ability of an interrogator which is subjective. For instance, a smart interrogator could detect a small grammatical error of one player's answer which helps him to determine that this player is not a human whereas another interrogator may not be able to notice. In this example, two interrogators will give a different answer which causes the result of the Turing Test to be inconsistent. What is the conclusion in this circumstance? Can this machine think?

Another point that's worth discussing is Turing proposed answering questions as a way to illustrate the ability to think. Is this a suitable evaluation? Human brains think all the time even when they are not answering questions, how Turing Test going to explain other kinds of thought like instinct. Is there any other method to show thinking ability? Moreover, the conversation-based test that Turing proposed has a limitation that both interrogator and players must have some knowledge in common(language, for example) otherwise the result is nonsense. 

\end{document}
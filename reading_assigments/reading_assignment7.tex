\documentclass{article}
\usepackage[utf8]{inputenc}
\usepackage{amsmath}
\usepackage{graphicx}
\usepackage{float}
\usepackage[font=scriptsize,labelfont=bf]{caption}

\title{COMP6231 : Reading Assignment 7}
\author{Thanakorn Panyapiang\\
(31446612, tp2n19@soton.ac.uk)}
\date{}

\begin{document}
\maketitle

Brooks proposed the new way of developing artificial intelligence system called behaviour-based artificial intelligence. Unlike the traditional approach where there is a single unit that possesses all data and solves problems for the whole system, the new approach is based on the principles that intelligence is composed of layers of independent but connected computational units where each unit can have a perceptual component and solve self-related problems. Sophisticated functionalities can be obtained by combining several components that have specific behavior together.\\
\indent In my perspective, the theory fits really well in many domains especially those regarding multi-channel input and output. However, it fails to explain intellectual capabilities such as understanding, language, and reasoning. These abilities have only a few input and output channels but the underlying computation is sophisticated which looks more suitable with the old artificial intelligence approach.

\end{document}
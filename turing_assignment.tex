\documentclass{article}
\usepackage[utf8]{inputenc}
\usepackage{amsmath}

\title{COMP6231 : Reading Assigment 1}
\author{Thanakorn Panyapiang(31446612)}

\begin{document}

\maketitle
\section*{Is the Turing Test a way to dodge the issue of what intelligence is altogether?}

Although the Turing Test proposes a systematic way to find the answer to the question, "Can machines think?", the proposed method still depends significantly on the intellectual ability of an interrogator which is subjective. For instance, a smart interrogator could detect a small grammatical error of one player's answer which helps him to determine that this player is not a human whereas another interrogator may not be able to notice. In this example, two interrogators will give a different answer which causes the result of the Turing Test to be inconsistent. What is the conclusion in this circumstance? Can this machine think?

Other questions:
\begin{itemize}
\item Is answering questions the only way to show thinking ability(humans think all the time even when they are not answering questions)? Is there any other way to illustrate it?
\item Are humans the only species that can think? Suppose that there is a machine that has a thought process resemble to an animal(dog, for example). If this machine is tested using Turing method and it fails, Is it fair to say that this machine cannot think?
\end{itemize}



\end{document}